\documentclass{article}
\usepackage{graphicx} % Required for inserting images
\usepackage{ctex}
\usepackage{amsmath} % for the equation* environment
\usepackage[a4paper, margin=0.75in]{geometry}
\usepackage{hyperref}

\title{SLAM技术习题第一章}

\author{Kun Lin}
\date{August 2023}

\begin{document}
\maketitle
\section{分别用左右扰动模型,计算$\frac{\partial R^{-1}p}{\partial R}$.}
首先进行一些必要的引理推导:
\begin{equation*}
\begin{aligned}
R &= \mathrm{exp}(\phi^{\wedge}) \\
\Rightarrow \phi^{\wedge} &= \mathrm{log}(R)\\
R^{-1}
    &= \mathrm{exp}(\phi'^{\wedge}) \\
\Rightarrow \phi'^{\wedge}
    &= \mathrm{log}(R^{-1}) \\
    &=-\mathrm{log}(R) \\
    &=-\phi^{\wedge} \\
\Rightarrow \phi'
    &= -\phi \\
\Rightarrow R^{-1} &= \mathrm{exp}(-\phi^{\wedge})
\end{aligned}
\end{equation*}

左扰动模型:设旋转小量为$\theta$,则其求导可得:

\begin{equation*}
\begin{aligned}
\frac{\partial R^{-1}p}{\partial \theta}
    &= \lim\limits_{\theta\to 0}
        \frac
            {\mathrm{exp(\theta^{\wedge})}\mathrm{exp(-\phi^{\wedge})}p - \mathrm{exp(-\phi^{\wedge})}p}
            {\theta} \\
    &\text{Using 1st order Taylor expansion} \\
    &\approx \lim\limits_{\theta\to 0}
        \frac
            {(I+\theta^{\wedge})\mathrm{exp(-\phi^{\wedge})}p - \mathrm{exp(-\phi^{\wedge})}p}
            {\theta} \\
    &= \lim\limits_{\theta\to 0}
        \frac
            {\theta^{\wedge}\mathrm{exp(-\phi^{\wedge})}p}
            {\theta} \\
    &= \lim\limits_{\theta\to 0}
        \frac
            {\theta^{\wedge}R^{-1}p}
            {\theta} \\
    &= \lim\limits_{\theta\to 0}
        \frac
            {-(R^{-1}p)^{\wedge}\theta}
            {\theta} \\
    &= -(R^{-1}p)^{\wedge}
\end{aligned}
\end{equation*}

右扰动模型:设旋转小量为$\theta$,则其求导可得:

\begin{equation*}
\begin{aligned}
\frac{\partial R^{-1}p}{\partial \theta}
    &= \lim\limits_{\theta\to 0}
        \frac
            {\mathrm{exp(-\phi^{\wedge})}\mathrm{exp(\theta^{\wedge})}p - \mathrm{exp(-\phi^{\wedge})}p}
            {\theta} \\
    &\text{Using 1st order Taylor expansion} \\
    &\approx \lim\limits_{\theta\to 0}
        \frac
            {\mathrm{exp(-\phi^{\wedge})}(I+\theta^{\wedge})p - \mathrm{exp(-\phi^{\wedge})}p}
            {\theta} \\
    &= \lim\limits_{\theta\to 0}
        \frac
            {\mathrm{exp(-\phi^{\wedge})}\theta^{\wedge}p}
            {\theta} \\
    &= \lim\limits_{\theta\to 0}
        \frac
            {-R^{-1}p^{\wedge}\theta}
            {\theta} \\
    &= -R^{-1}p^{\wedge}
\end{aligned}
\end{equation*}

\section{分别用左右扰动模型,计算$\frac{\partial R_1R_2^{-1}}{\partial R_2}$.}

左扰动模型:设旋转小量为$\theta$,则其求导可得:

\begin{equation*}
\begin{aligned}
\frac{\partial \mathrm{Log}(R_1R_2^{-1})}{\partial R_2}
    &= \lim\limits_{\theta\to 0}
        \frac
            {\mathrm{Log}(R_1\mathrm{Exp}(\theta)R_2^{-1})-\mathrm{Log}(R_1R_2^{-1})}
            {\theta} \\
    &\text{Using adjoint map} \\
    &= \lim\limits_{\theta\to 0}
        \frac
            {\mathrm{Log}(R_1R_2^{-1}R_2\mathrm{Exp}(\theta)R_2^{-1})-\mathrm{Log}(R_1R_2^{-1})}
            {\theta} \\
    &= \lim\limits_{\theta\to 0}
        \frac
            {\mathrm{Log}(R_1R_2^{-1}\mathrm{Exp}({R_2\theta}))-\mathrm{Log}(R_1R_2^{-1})}
            {\theta} \\
    &\text{Using BCH approximation} \\
    &= \lim\limits_{\theta\to 0}
        \frac
            {\mathrm{Log}(R_1R_2^{-1})+J_r^{-1}(\mathrm{Log}(R_1R_2^{-1}))\mathrm{Log}(\mathrm{Exp(R_2\theta)})-\mathrm{Log}(R_1R_2^{-1})}
            {\theta} \\
    &= J_r^{-1}(\mathrm{Log}(R_1R_2^{-1}))R_2
\end{aligned}
\end{equation*}

右扰动模型:设旋转小量为$\theta$,则其求导可得:

\begin{equation*}
\begin{aligned}
\frac{\partial \mathrm{Log}(R_1R_2^{-1})}{\partial R_2}
    &= \lim\limits_{\theta\to 0}
        \frac
            {\mathrm{Log}(R_1R_2^{-1}\mathrm{Exp}(\theta))-\mathrm{Log}(R_1R_2^{-1})}
            {\theta} \\
    &\text{Using BCH approximation} \\
    &= \lim\limits_{\theta\to 0}
        \frac
            {\mathrm{Log}(R_1R_2^{-1})+J_r^{-1}(\mathrm{Log}(R_1R_2^{-1}))\theta-\mathrm{Log}(R_1R_2^{-1})}
            {\theta} \\
    &= J_r^{-1}(\mathrm{Log}(R_1R_2^{-1}))
\end{aligned}
\end{equation*}

\section{将实践环节中的运动学修改成带一定角速度的平抛运动。车辆受固定的$Z$轴角速度影响,具有一定的初始水平速度,同时受$-Z$方向的重力加速度影响。请修改程序,给出动画演示。}

代码diff位于\href{https://github.com/kunlin596/slam_in_autonomous_driving/commit/94c7c97492d4d68331b6ec1dd58b313b7d00a28a}{此处},
完整文件位于\href{https://github.com/kunlin596/slam_in_autonomous_driving/blob/homework-1/src/ch2/motion.cc}{motion.cc},
演示动画位于\href{https://github.com/kunlin596/slam_in_autonomous_driving/blob/homework-1/homework/homework1_motion.m4v}{homework1\_motion.m4v}
。

\section{自行寻找相关材料,说明高斯牛顿法和 Levenberg-Marquardt 法在处理非线性迭代时的差异。}

高斯牛顿法和 LM 法都是用于求解非线性最小二乘问题的方法,其区别在于 LM 法在高斯牛顿法的基础上加入了一个阻尼系数,用于控制迭代过程中的步长,从而使得 LM 法具有更好的收敛性。在每一次的迭代之后,通过计算更新新区域的半径(阻尼系数),使得 LM 在梯度下降和牛顿法之间自动选择(使用一次或者二次函数拟合),从而使得 LM 法具有更好的收敛性。

\end{document}
